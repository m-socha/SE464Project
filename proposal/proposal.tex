  \documentclass[12pt]{article}

  \usepackage[fleqn]{amsmath}
  \usepackage{color}
  \usepackage{listings}
  \usepackage[margin=1in]{geometry}
  \usepackage{indentfirst}

  \begin{document}

  \begin{center}
  \vspace*{\fill}
  {\Large\bf University of Waterloo}\\
  \vspace{3mm}
  {\large\bf SE464 Fall 2017}\\
  \vspace{3mm}
  {\Large\bf Project Proposal - WatNotes}\\
  \vspace{5mm}
  Michael Socha, Mitchell Kember, Do Gyun Kim, Myungheon Chun\\
  \vspace{3mm}
  msocha@edu.uwaterloo.ca, mkember@edu.uwaterloo.ca, dg3kim@edu.uwaterloo.ca, m5chun@edu.uwaterloo.ca\\
  \vspace*{\fill}
  \end{center}

  \newpage

  \section{Project Overview}
  Our project is called WatNotes, which is a note sharing platform for students at the University of Waterloo. WatNotes will provide students with a platform to upload their notes, tag them by course and topic, and share them with other students at the University. The tagging of uploaded notes will allow for an effective course-centric search system for finding other students' notes. WatNotes will also allow students to edit uploaded notes (e.g. by adding inline comments), allowing them to refine their notes in a collaborative environment. Although the prototype of WatNotes is targeted specifically for the University of Waterloo, WatNotes is designed to be extensible to other universities or even different types of organizations. \\

  There are existing platforms that facilitate note sharing on a large scale, such as Microsoft OneNote and Google Photos. Likewise, students often already share notes on a small scale within their circle of friends. However, there is a lack of moderate-scale note sharing platforms that bring together people with common learning interests. WatNotes attempts to fill this gap by facilitating the sharing of notes between students learning similar topics who would otherwise likely not meet. \\

  The main reason WatNotes is interesting is that it attempts to solve a real problem at the University of Waterloo. Students who miss class or would like to look over another student's notes will be able to do so in a few clicks instead of making a post on Piazza asking for notes or aimlessly asking random students. Likewise, students who would like to share their notes will be able to reach people beyond their immediate group of friends, since content on WatNotes will be centered around course and subject rather than its authors. \\

WatNotes will be available as both a website as well as a mobile (Android) application. Android was the chosen mobile platform due to its popularity and rapid release cycle, making it good for initial app prototypes. Supporting some sort of mobile platform is critical for the platform's growth, since note uploads are simpler and less time-consuming to perform with mobile devices than through desktop applications. Moreover, for short sessions or casual note searching, accessing WatNotes on a mobile device is the natural choice for users, since mobile sessions are quicker to start and require less commitment and attention compared to desktop sessions. \\

While certain functions of WatNotes can be performed on mobile devices, other actions such as note studying and collaboration require in-depth concentration from users over long sessions, making a desktop environment essential for WatNotes. Cross-platform support for WatNotes is necessary to allow the majority of the University of Waterloo's student body to access the platform, so the desktop environment will be implemented as a website. \\
\newpage

\section{Project Properties}
\subsection{Functional Properties}
  \subsubsection{Uploading and Sharing}
  Students should be able to upload their existing notes to WatNotes, making them public and instantly sharing them with all other students. If they are taking pictures of written notes, for example, uploading via the mobile app may be most convenient. The system should accommodate the user by accepting a wide range of formats, at minimum including text files, PDF documents, and images. The goal is to give students the flexibility to express their particular note-taking style, not to enforce a rigid format. Furthermore, students should be able to export any set of notes and save them locally; they should not feel locked into the WatNotes platform.
  \subsubsection{Editing and Collaborating}
  Once a student has uploaded notes, they should be able to make edits and corrections on the website or mobile app rather than re-uploading fixed notes. In fact, the editing features should be powerful enough that students could simply create notes inside WatNotes in the first place if they wanted to. We expect both methods to be used, and we will support both. Students can edit by typing text, adding images, and possibly by other methods such as entering equations or blocks of code: we intentionally leave this open-ended. In addition, any student should be able to add a comment attached to a specific point in the notes. This feature may be used to simply point out errors, but it may also be used to have a public discussion that adds value to the notes and clarifies them.
  \subsubsection{Discovering}
  Notes on WatNotes should be easily discoverable. Students should be able to search for course notes and for specific content within notes using a global search feature that considers course names, topics, keywords that appear in notes, and names of uploaders. Students who upload notes should be able to add navigation features such as headings and tags to further improve navigation and discoverability. If a student finds a set of notes for a particular course, they should be able to instantly access all other sets of notes for that course uploaded by other people. Finally, when searching for notes, the highest quality notes should appear first, where quality is determined by some heuristics or by user upvoting/downvoting.
\subsection{Example Use Scenarios}
  \subsubsection{Student Misses Lecture}
  Consider a student who has missed an important lecture due to a co-op interview. The student asks a few friends for their lecture notes, but they are sloppy, incomplete, and difficult to understand. Next, the student makes a post on Piazza asking for notes from the previous lecture, but the request just sits there for hours with no reply. The student could ask random people in the class for their notes, but this  is inconvenient, and most notes are unlikely to be very high quality anyway. \\

  With WatNotes, a student simply searches for the course they missed and can find relevant lecture notes within seconds. The available sets of notes have been refined in a collaborative environment, and hence are likely to be understandable and clear. Moreover, since several different sets of notes are likely to be available for the missed lecture, the student can use one set of notes to fill any omissions another one may have.
  \subsubsection{Student Wants to Release Notes into Community}
  An altruistic student would like to release their notes to the general community, but lacks a good platform with which to do so. Posts on social media platforms such as Facebook are likely to get lost amid other content, and few people on these types of platforms are looking for notes anyway. With WatNotes, the student can upload their notes and release them to the entire university on a platform designed solely for the purpose of note sharing. On top of helping many people the student would not have otherwise reached, other students might collaboratively edit the notes and add inline comments to point out inaccuracies or omissions, helping the uploader in their studying as well.
\subsection{Non-Functional Properties}
  \subsubsection{Efficiency}
  Our group will have limited resources to spend on an efficient server with high performance, so we will have to use free resources which may not be as powerful. However, students would still expect a good performance from WatNotes regardless of which resources are used. The system will have to meet the performance needs of users or else very few students will be inclined to use WatNotes.

  \subsubsection{Dependability}
  As users comment and collaborate on notes, they want reassurance that the work they are committing is saved and retrievable. The system might cause a lot of pain for students if all of their comments on a document are lost when they forget to save before losing access to wifi or closing the application. That is why the system's dependability, specifically the fault tolerance of the system, will be very important. WatNotes should constantly save the work that people do on notes and even have some offline capabilities in case of Internet disconnections. 

  \subsubsection{Evolvability}
  The system will have to be easily extendable to changes in the specification. Student needs and use cases of WatNotes may change or new features might be required. Any changes brought to the codebase to satisfy the new specification should require little refactoring. Designing an architecture with evolvability in mind is particularly important for an open-ended project such as WatNotes.
\end{document}
