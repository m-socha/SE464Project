  \documentclass[12pt]{article}

  \usepackage[fleqn]{amsmath}
  \usepackage{color}
  \usepackage{listings}
  \usepackage[margin=1in]{geometry}
  \usepackage{indentfirst}

  \begin{document}

  \begin{center}
  \vspace*{\fill}
  {\Large\bf University of Waterloo}\\
  \vspace{3mm}
  {\large\bf SE464 Fall 2017}\\
  \vspace{3mm}
  {\Large\bf Project Proposal - WatNotes}\\
  \vspace{5mm}
  Michael Socha, Mitchell Kember, Do Gyun Kim, Myungheon Chun\\
  \vspace{3mm}
  msocha@edu.uwaterloo.ca, mkember@edu.uwaterloo.ca, dg3kim@edu.uwaterloo.ca, m5chun@edu.uwaterloo.ca\\
  \vspace*{\fill}
  \end{center}

  \newpage

  \section{Project Overview}
  Our project is called WatNotes, which is a note sharing platform for students at the University of Waterloo. WatNotes will provide students with a platform to upload their notes, tag them by course and topic, and share them with other students at the University. The tagging of uploaded notes will allow for an effective course-centric search system for finding other students' notes. WatNotes will also allow students to edit uploaded notes (e.g. by adding inline comments), allowing them to refine their notes in a collaborative environment. Although the prototype of WatNotes is targeted specifically for the University of Waterloo, WatNotes is designed to be extensible to other universities or even different types of organizations. \\

  There are existing platforms that facilitate note sharing on a large scale, such as Microsoft OneNote and Google Photos. Likewise, students often already share notes on a small scale within their circle of friends. However, there is a lack of moderate-scale note sharing platforms that bring together people with common learning interests. WatNotes attempts to fill this gap by facilitating the sharing of notes between students learning similar topics who would otherwise likely not meet. \\

  The main reason WatNotes is interesting is that it attempts to solve a real problem at the University of Waterloo. Students who miss class or would like to look over another student's notes will be able to do so in a few clicks instead of making a post on Piazza asking for notes or aimlessly asking random students. Likewise, students who would like to share their notes will be able to reach people beyond their immediate group of friends, since content on WatNotes will be centered around course and subject rather than its authors. \\

WatNotes will be available as both a website as well as a mobile (Android) application. Android was the chosen mobile platform due to its popularity and rapid release cycle, making it good for initial app prototypes. Supporting some sort of mobile platform is critical for the platform's growth, since note uploads are simpler and less time-consuming to perform with mobile devices than through desktop applications. Moreover, for short sessions or casual note searching, accessing WatNotes on a mobile device is the natural choice for users, since mobile sessions are quicker to start and require less commitment and attention compared to desktop sessions. \\

While certain functions of WatNotes can be performed on mobile devices, other actions such as in-depth note studying and collaboration require in-depth concentration from users over long sessions, making a desktop environment essential for WatNotes. Cross-platform support for WatNotes is necessary to allow the majority of the University of Waterloo's student body to access the platform, so the desktop environment will be implemented as a website. \\
\newpage

\section{Project Properties}
\subsection{Functional Properties}
\subsection{Example Use Scenarios}
  \subsubsection{Student Misses Lecture}
  Consider a student who has missed an important lecture due to a co-op interview. The student asks a few friends for their lecture notes, but they are sloppy, incomplete, and difficult to understand. Next, the student makes a post on Piazza asking for notes from the previous lecture, but the request just sits there for hours with no reply. The student could ask random people in the class for their notes, but this  is inconvenient, and most notes are unlikely to be very high quality anyway. \\

  With WatNotes, a student simply searches for the course they missed and can find relevant lecture notes within seconds. The available sets of notes have been refined in a collaborative environment, and hence are likely to be understandable and clear. Moreover, since several different sets of notes are likely to be available for the missed lecture, the student can use one set of notes to fill any omissions another one may have.
  \subsubsection{Student Wants to Release Notes into Community}
  An altruistic student would like to release their notes to the general community, but lacks a good platform with which to do so. Posts on social media platforms such as Facebook are likely to get lost amid other content, and few people on these types of platforms are looking for notes anyway. With WatNotes, the student can upload their notes and release them to the entire university on a platform designed solely for the purpose of note sharing. On top of helping many people the student would not have otherwise reached, other students might collaboratively edit the notes and add inline comments to point out inaccuracies or omissions, helping the uploader in their studying as well.
\subsection{Non-Functional Properties}

\end{document}
