  \documentclass[12pt]{article}

  \usepackage[fleqn]{amsmath}
  \usepackage{color}
  \usepackage{listings}
  \usepackage[margin=1in]{geometry}
  \usepackage{indentfirst}
  \usepackage{graphicx}
  \usepackage{float}

  \begin{document}

  \begin{center}
  \vspace*{\fill}
  {\Large\bf University of Waterloo}\\
  \vspace{3mm}
  {\large\bf SE464 Fall 2017}\\
  \vspace{3mm}
  {\Large\bf Project Status Report and Demo Description}\\
  \vspace{5mm}
  {\Large Project Name: WatNotes}\\
  \vspace{5mm}
  Michael Socha, Mitchell Kember, Do Gyun Kim, Myungheon Chun\\
  \vspace{3mm}
  msocha@edu.uwaterloo.ca, mkember@edu.uwaterloo.ca, dg3kim@edu.uwaterloo.ca, m5chun@edu.uwaterloo.ca\\
  \vspace*{\fill}
  \end{center}

  \newpage

  \section{Status Report}
    WatNotes is a platform designed to allow students at the University of Waterloo to effectively upload, share
    and collaboratively edit notes. WatNotes has a mobile client used primarily for uploading notes and searching
    through other user notes. WatNotes also supports a website, which is used for activities that require greater
    focus such as in-depth studying and collaboration. \\

    Since the prototype demo, WatNotes' development team has managed to implement all key functional properties described
    by the project's prescriptive architecture. In particular, the following features are supported:
    
    \begin{itemize}
      \item
        \textbf{Uploading and sharing notes}. As originally intended, student can use WatNotes' mobile client to upload existing
        files of notes (e.g. .txt files or images), or to take pictures of notes and upload the resulting files. Notes are
        automatically public, and hence viewable by other WatNotes users.
      \item
        \textbf{Editing and collaboration}. WatNotes' web client supports adding comments on notes to support a collaborative environment.
        Students can leave comments to ask questions or point out inaccuracies in notes, improving overall note quality for
        the platform's users.
      \item
        \textbf{Search and discoverability}. WatNotes' platform allows users to search notes uploaded to the platform based on their content.
        Such search functionality makes the sharing aspect of WatNotes more effective, since students are able to easily find the
        material they want to study. Moreover, WatNotes' web platform supports search across content beyond notes, including notebooks
        (collections of notes), users, and comments.
    \end{itemize}

    Certain extension features, such as note ratings and parsing notes for custom formats (e.g. a Latex parser) remain unimplemented.
    The majority of effort in this project focused on implementing and refining the key fuctional requirements - tagging on extension
    features without a solid foundation did not make sense for the the product. \\

    Another suggested feature that did not get implemented is a UW-specific login system (i.e. integration with Quest's login).
    Based on our research, there does not seem to be a public API for UWaterloo-wide login. Moreover, this project can be extended
    to other universities, in which case a generic login system would likely be preferable.
  \newpage

  \section{Demo Description}
    The demo will open with a high-level overview of WatNotes, including a general description of the features supported by the system's web
    and mobile platforms. The presentation will then briefly cover some of the architectural and technical details of each platform. \\
    
    After that, the presentation will cover sample use cases of the platform. The main focus will be on the example use cases
    described in the project's initial proposal. These include:

    \begin{itemize}
      \item
        A student has missed a lecture and would like to read its notes. WatNotes allows this student to perform a simple search on
        its mobile platform using keywords of the missed lecture, and returns a list of all relevant notes. The student is now able
        to read through the notes and catch up on the missed lecture.
      \item
        A student would like to make their notes public for the good of the University community. The student uses WatNotes' mobile
        application to upload files of their notes, which are automatically made public to other users of the platform. Later, a student
        studying using WatNotes' website encounters these notes, and points out a minor inaccuracy. Thus, the uploader was able to help
        the community by releasing their notes, and was also benefited by other users helping improve their notes in WatNotes' collaborative
        environment.
    \end{itemize}

    The demo will conclude with a Q\&A session.

\end{document}
