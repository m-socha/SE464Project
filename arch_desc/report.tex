  \documentclass[12pt]{article}

  \usepackage[fleqn]{amsmath}
  \usepackage{color}
  \usepackage{listings}
  \usepackage[margin=1in]{geometry}
  \usepackage{indentfirst}
  \usepackage{graphicx}
  \usepackage{float}

  \begin{document}

  \begin{center}
  \vspace*{\fill}
  {\Large\bf University of Waterloo}\\
  \vspace{3mm}
  {\large\bf SE464 Fall 2017}\\
  \vspace{3mm}
  {\Large\bf Project Architecture and Design}\\
  \vspace{5mm}
  {\Large Project Name: WatNotes}\\
  \vspace{5mm}
  Michael Socha, Mitchell Kember, Do Gyun Kim, Myungheon Chun\\
  \vspace{3mm}
  msocha@edu.uwaterloo.ca, mkember@edu.uwaterloo.ca, dg3kim@edu.uwaterloo.ca, m5chun@edu.uwaterloo.ca\\
  \vspace*{\fill}
  \end{center}

  \newpage

  \section{Project Architecture}
  \subsection{Overview}
    WatNotes is a platform designed to allow students at the University of Waterloo to effectively upload, share
    and collaboratively edit notes. WatNotes has a mobile client used primarily for uploading notes, searching for
    other relevant notes, and other activities that tend to require little user commitment and have short session times.
    A website it also supported for activities that require greater user commitment and focus, such as studying and
    collaboratively editing notes. \\

    Once the supported platforms for WatNotes were finalized, development was divided into three separate teams, namely
    backend, mobile, and web. The backend team is run by Mitchell Kember, and is responsible for developing WatNotes' API
    endpoints, infrastructure and data storage. The mobile team is run by Michael Socha, and is responsible for developing
    WatNotes' Android application. The web team is run jointly by Myungheon Chun and Do Gyun Kim, and is responsible for
    developing and deploying the Watnotes' website. \\

    In this section, WatNotes' architecture is analyzed in a way that treats the backend, mobile client, and website of the system
    somewhat as black boxes (low level of detail). The subsequent sections expand upon each of these platforms in greater detail. \\

    A high-level component diagram where each platform is described as a separate component is described below. \\

    From a high-level overview where each of WatNotes' platforms are treated as separate components, the main architectural styles applied are
    a client-server architecture and a blackboard-based shared memory architecture. A client-server architecture is built with Watnotes' mobile
    applications and website as the clients, which are communicating with WatNotes' server through a REST API. A blackboard-based shared memory
    architecture is achieved through the system-wide data store in the backend, to which clients can upload and retrieve data, maintaining a
    global system state. \\

  \subsection{Backend}
  \subsection{Android Application}
  \subsection{Website}

  \newpage

  \section{Project Design}
  \subsection{Overview}
  \subsection{Backend}
  \subsection{Android Application}
  \subsection{Website}

\end{document}
