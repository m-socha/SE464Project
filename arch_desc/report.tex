  \documentclass[12pt]{article}

  \usepackage[fleqn]{amsmath}
  \usepackage{color}
  \usepackage{listings}
  \usepackage[margin=1in]{geometry}
  \usepackage{indentfirst}
  \usepackage{graphicx}
  \usepackage{float}

  \begin{document}

  \begin{center}
  \vspace*{\fill}
  {\Large\bf University of Waterloo}\\
  \vspace{3mm}
  {\large\bf SE464 Fall 2017}\\
  \vspace{3mm}
  {\Large\bf Project Architecture and Design}\\
  \vspace{5mm}
  {\Large Project Name: WatNotes}\\
  \vspace{5mm}
  Michael Socha, Mitchell Kember, Do Gyun Kim, Myungheon Chun\\
  \vspace{3mm}
  msocha@edu.uwaterloo.ca, mkember@edu.uwaterloo.ca, dg3kim@edu.uwaterloo.ca, m5chun@edu.uwaterloo.ca\\
  \vspace*{\fill}
  \end{center}

  \newpage

  \section{Project Architecture}
  \subsection{Overview}
    WatNotes is a platform designed to allow students at the University of Waterloo to effectively upload, share
    and collaboratively edit notes. WatNotes has a mobile client used primarily for uploading notes, searching for
    other relevant notes, and other activities that tend to require little user commitment and have short session times.
    A website it also supported for activities that require greater user commitment and focus, such as studying and
    collaboratively editing notes. \\

    Once the supported platforms for WatNotes were finalized, development was divided into three separate teams, namely
    backend, mobile, and web. The backend team is run by Mitchell Kember, and is responsible for developing WatNotes' API
    endpoints, infrastructure and data storage. The mobile team is run by Michael Socha, and is responsible for developing
    WatNotes' Android application. The web team is run jointly by Myungheon Chun and Do Gyun Kim, and is responsible for
    developing and deploying the Watnotes' website. \\

    In this section, WatNotes' architecture is analyzed in a way that treats the backend, mobile client, and website of the system
    somewhat as black boxes (low level of detail). The subsequent sections expand upon each of these platforms in greater detail. \\

    A high-level component diagram where each platform is described as a separate component is described below. \\

    From a high-level overview where each of WatNotes' platforms are treated as separate components, the main architectural styles applied are
    a client-server architecture and a blackboard-based shared memory architecture. A client-server architecture is built with Watnotes' mobile
    applications and website as the clients, which are communicating with WatNotes' server through a REST API. A blackboard-based shared memory
    architecture is achieved through the system-wide data store in the backend, to which clients can upload and retrieve data, maintaining a
    global system state. \\

  \subsection{Backend}
  \subsection{Android Application}
  \subsection{Website}

  \newpage

  \section{Project Design}
  \subsection{Overview}
  \subsection{Backend}
  \subsection{Android Application}
    This section serves to provide a general overview of the Android application's class-level design. All of the described components run on the Android
    device on which the application is operating (no distributed logic). \\

    One major design decision of the app concerns how the application handles Android's activity lifecycle. In short, activities are the basic building blocks of
    an Android app, providing a user with a UI, resource management and handling certain OS events. Much of this logic is common to all activities, and
    hence is abstracted into a common BaseActivity class. BaseActivity is an abstract class that follows a template design pattern, with functionality
    such as getting the UI layout container (getLayoutId) and initializing the UI once it is loaded (setupUi) implemented by concrete subclasses. DrawerActivity
    is an abstract subclass that extends from BaseActivity, and implements a side navigation drawer - most classes in WatNotes inherit from DrawerActivity. \\

    Each activity must implement the createUiFragment and createServiceFragment methods. UiFragments implement an activity's UI (e.g. loading and
    initializing layout resources), while ServiceFragments are responsible for handling an activity's networking by providing an interface
    to service instances (adapter design patter). The BaseActivity attaches both of these fragments to the activity, where they can have independent sub-lifecycles
    within the activity. UiFragment and ServiceFragment are both abstract base classes, with concrete UiFragment and ServiceFragment associated with each
    concrete class. \\

    Another major design decision concerns how the app's networking layer is organized. An abstract base class called ApiRequest defined from which the abstract
    SingleApiService and MultiApiService classes extend. SingleApiService classes are services for which there is never expected to be more than one active request. Thus, whenever
    a service is started through a call to startRequest(), any existing service is canceled. On the other hand, MultiApiServices may have multiple requests active,
    and a request is only canceled in calls to startRequest() if it shares the requestId of the new request. Concrete ApiRequest classes extend from either
    SingleApiService or MultiApiService classes, following a template design pattern. \\
  \subsection{Website}
  The front-end of WatNotes is a react-redux application that runs on a node server and uses webpack for dependency management. The source code for the project is under the src directory. In src, the main index.js file can be found which initializes the react router and redux store. Other important subdirectories in the src directory include store, containers, and components. Any new dependencies required in the project are handled using npm and should be added to the webpack build as required (look in the webpack directory).\\

  The react router implements the logic for deciding which components to render based on the url. When the browser goes to a specific url, a container is chosen to render. The react router is an example of the strategy design pattern. Containers define the data that will be passed down to components to render. They define the data to be passed down by referencing the redux store and are responsible for actions such as fetching data from an api and bootstraping the components with the fetched data. This is an example of one way directional data flow where components only receive data and cannot modify the data received. This pattern decreases complexity since the developer only has to manage 1 data state (the redux store) and can think of components as only having the responsibility of rendering the ui (single responsibility principle). A component is a class that represents what you see visually only. Components are designed to be reusable and has no dependencies other than the properites passed into the class constructor. This minimizes coupling as ui components have no dependencies and it also maximizes flexibility which helps accommodate for future requirement changes. If a new ui component or an existing requirement for a ui component is changed, minimal changes are needed since the UI would be broken up into small components.\\

  The use of react and redux separates the the ui logic from the data state logic. Redux containers can fetch data using the backend api and pass it down to the components. As more features are added, more properties will have to be added to the state along with more containers to decide how to pass down those new properties down to components. Since there is a single store, adding more data to the store will not affect other parts of the application.

\end{document}
